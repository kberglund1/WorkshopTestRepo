\documentclass{ximera}
%\usepackage{unicode-math}
%\usepackage{hyperref}

\title{Math tagged with MathML Structure Elements}
\author{LaTeX Team}
\begin{document}

\begin{abstract}
    This is an abstract.
\end{abstract}
\maketitle

\section{Basic mathematical expressions}

If $x$ is real, then $x^{2} < 0$. Is this Planck's constant? $h$.

A matrix equation.
\[
\begin{pmatrix}0&1\\1&0\end{pmatrix}
\begin{pmatrix}a&b\\c&d\end{pmatrix}
=
\begin{pmatrix}c&d\\a&b\end{pmatrix}
\]

\section{The Lorenz Equations}
\[\begin{aligned}
\dot{x} & = \sigma(y-x) \\
\dot{y} & = \rho x - y - xz \\
\dot{z} & = -\beta z + xy
\end{aligned} \]

\section{The Cauchy-Schwarz Inequality}
\[ \left( \sum_{k=1}^n a_k b_k \right)^2 \leq \left( \sum_{k=1}^n a_k^2 \right) \left( \sum_{k=1}^n b_k^2 \right) \]
\section{A Cross Product Formula}
\[\mathbf{V}_1 \times \mathbf{V}_2 =  \begin{vmatrix}
\mathbf{i} & \mathbf{j} & \mathbf{k} \\
\frac{\partial X}{\partial u} &  \frac{\partial Y}{\partial u} & 0 \\
\frac{\partial X}{\partial v} &  \frac{\partial Y}{\partial v} & 0
\end{vmatrix}  \]

The probability of getting $k$ heads when flipping $n$ coins is: 
\[P(E) = {n \choose k} p^k (1-p)^{ n-k} \]

\section{An Identity of Ramanujan}
\[ \frac{1}{\Bigl(\sqrt{\phi \sqrt{5}}-\phi\Bigr) e^{\frac25 \pi}} =
1+\frac{e^{-2\pi}} {1+\frac{e^{-4\pi}} {1+\frac{e^{-6\pi}}
{1+\frac{e^{-8\pi}} {1+\ldots} } } } \]

\section{A Rogers-Ramanujan Identity}
\[  1 +  \frac{q^2}{(1-q)}+\frac{q^6}{(1-q)(1-q^2)}+\cdots =
\prod_{j=0}^{\infty}\frac{1}{(1-q^{5j+2})(1-q^{5j+3})},
\quad\quad \text{for} |q|<1. \]

\section{Maxwell's Equations}
\[  \begin{aligned}
\nabla \times \vec{\mathbf{B}} -\, \frac1c\, \frac{\partial\vec{\mathbf{E}}}{\partial t} & = \frac{4\pi}{c}\vec{\mathbf{j}} \\
\nabla \cdot \vec{\mathbf{E}} & = 4 \pi \rho \\
\nabla \times \vec{\mathbf{E}}\, +\, \frac1c\, \frac{\partial\vec{\mathbf{B}}}{\partial t} & = \vec{\mathbf{0}} \\
\nabla \cdot \vec{\mathbf{B}} & = 0 
\end{aligned}\]

\section{Links}
\href{http://mslc.osu.edu}{
The MSLC} is a great place to work on homework!

\end{document}