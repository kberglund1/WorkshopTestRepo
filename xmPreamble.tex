%
% Add here extra packages/macro's 
% that are  *automatically* loaded by all documents of class 'ximera' or 'xourse'
%
\typeout{Start loading xmPreamble.tex}         % Write to logfile (for debugging !)

%%
%% Set defaults
%%
\license{CC: 0}                    % Default license
\author{Ximera Development Team}   % Default author  (if one is wanted ...)  

%%
%%  Load extra packages
%%
%%  NOTE: NOT ALL PACKAGES ARE PRESENT in the docker containers
%%   ( use a container of type '-full' to get a FULL TeXLive )

\usepackage{xstring}     % Needed (at least) infra  for \IfSubStr
\usepackage{unicode-math}
\usepackage{hyperref}

%%
%%  Define extra macro's
%%  Be careful with layout-macro's:
%%    - they might not work/confuse the HTML version
%%    - if needed, they could be put in xmPrintStyle.sty   
%%
% \newcommand{\R}{\mathbb{R}


% 
% This is a rather very advanced example of using TeX-functionality
%  First-time users should just skip reading this.
%  When using this xmPreamble for your own repo,
%  remove this section, UNLESS YOU WANT pre-2025-OSU versions of hints and expandables
%  If so, KEEP ONLY the two lines with \def\xmNot....
%
%  With this code fragment, one gets BY DEFAULT OSU-style hints.
%  UNLESS the filename contains the string 'Variant', 
%         when one gets the newer expandable hints. 
%  Once the OSU server gets updated, all this will become obsolete.
\IfSubStr{\jobname}{\detokenize{Variant}}{
    \typeout{Compiling NEW version for \jobname\ (with hint as expandable)}%
}{
    \typeout{Compiling OSU version of hints for \jobname}%
    % These variables give the OLD types of hints.
    % In the (near) future, this should no longer be needed.
    %  Only for use with OSU server, and xake > v2.1.3
    \def\xmNotHintAsExpandable{1}
    \def\xmNotExpandableAsAccordion{1}
}
% EndOfRatherVeryAdvancedExample